\documentclass{article}
\usepackage[utf8]{inputenc}
\usepackage[T1]{fontenc}
\usepackage{listings}
\usepackage{verbatim}
\usepackage[left=2cm,right=2cm,top=2cm,bottom=2cm]{geometry}
\title{Bonnes Pratiques en C}
\author{Prérequis: bases en langage C}
\date{}

\begin{document}
\maketitle
\section*{Macros de préprocesseur}
\begin{itemize}
\item inclusions
\item constantes
\item remplacements de texte (alias)
\item conditions
\item macros prédéfinies:
\end{itemize}

    \begin{lstlisting}[xleftmargin=\parindent]
    #ifndef MON_FICHIER_H
    #define ...
    #endif
    \end{lstlisting}

\begin{itemize}
    \item fonctions / avec nobre indéfini de variables
    \item concaténations
    \item affichage littéral
  \end{itemize}

\section*{Les essentiels}
\begin{itemize}
  \item \begin{verbatim}
    #include <stdlib.h>
  \end{verbatim}

  \item \begin{verbatim}
    #include <stdio.h>
  \end{verbatim}

  \item \begin{verbatim}
    #include <string.h>
  \end{verbatim}

  \item \begin{verbatim}
    #include <errno.h>
  \end{verbatim}
    => perror("..."); affiche ce qui est passé + un msg detaillé selon ERRNO
  \end{itemize}

  \section*{Macros}
  \begin{verbatim}
  #define PRINTVAR(x) fprintf(stderr, "DEBUG -- %s:%d" #x " =%d\n", _FILE_, _LINE_, x);
\end{verbatim}
  \begin{itemize}
    \item \#x recupère le litteral cad le "nom" de la variable
    \item affichage de la valeur de la variable pour un fichier et une	ligne donnée (celle de l'appel a PRINTVAR() ).
  \end{itemize}

\end{document}
